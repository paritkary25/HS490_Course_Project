\documentclass{article}
\usepackage{graphicx}
\usepackage[utf8]{inputenc}
\usepackage{epsfig}
\usepackage{float,graphicx}
\usepackage{listings}
\usepackage{xcolor}
\usepackage{hyperref}
\usepackage{amssymb}
\usepackage{amsmath}
\usepackage{pdfpages}
\usepackage{indentfirst}

\definecolor{codegreen}{rgb}{0,0.6,0}
\definecolor{codegray}{rgb}{0.5,0.5,0.5}
\definecolor{codepurple}{rgb}{0,0,0}
\definecolor{backcolour}{rgb}{1,1,1}

\hypersetup{
 colorlinks,
 citecolor=blue,
 linkcolor=blue,
 urlcolor=blue}

\lstdefinestyle{mystyle}{
    backgroundcolor=\color{backcolour},
    commentstyle=\color{codegreen},
    keywordstyle=\color{black},
    numberstyle=\tiny\color{codegray},
    stringstyle=\color{codepurple},
    basicstyle=\ttfamily\footnotesize,
    breakatwhitespace=false,
    breaklines=true,
    captionpos=b,
    keepspaces=true,
    numbers=left,
    numbersep=5pt,
    showspaces=false,
    showstringspaces=false,
    showtabs=false,
    tabsize=2
}

\lstset{style=mystyle}
\title{\textbf{Reflection of National Culture in Civil Hospital, Jalna}\\
HS490: Course Project}

\author{Yash Paritkar\\
Roll Number: 210070096\\
\\
Instructor: Prof. Pooja Purang}

\date{05 April, 2023}

\begin{document}
\maketitle

\newpage
\section*{Abstract}
\addcontentsline{toc}{section}{Abstract}
This report explores the reflection of national culture on Indian government organizations. The study draws on Hofstede's dimensions of culture to analyze the impact of cultural values on the functioning of Indian government organizations. Specifically, the report examines the dimensions of power distance, individualism vs. collectivism, masculinity vs. femininity, uncertainty avoidance, and long-term vs. short-term orientation.

The findings of the study indicate that Indian government organizations reflect the cultural values of the society they operate in. Power distance is high in Indian culture, and this is reflected in the hierarchical structure of government organizations. Collectivism is also a significant cultural value in India, and this is reflected in the emphasis on teamwork and group decision-making in government organizations.

Masculinity and femininity were found to be relatively balanced in Indian culture, and this is reflected in the diverse roles of men and women in government organizations. Uncertainty avoidance was found to be high in Indian culture, and this is reflected in the preference for established rules and procedures in government organizations. Finally, long-term orientation was found to be a significant cultural value in India, and this is reflected in the emphasis on long-term planning and policy-making in government organizations.

Overall, the findings suggest that the reflection of national culture on Indian government organizations is significant and should be taken into account in organizational decision-making and management. %The report concludes with recommendations for how government organizations can leverage cultural values to improve their performance and effectiveness.

\newpage
\section*{Acknowledgement}
\addcontentsline{toc}{section}{Acknowledgement}
I want to start by expressing my gratitude to Prof. Pooja Purang for giving me this chance and guiding me while I was writing this report. I also want to thank my father, Dr. Paritkar, for reviewing the survey's questionnaire and assisting me conduct it. I want to express my gratitude to Sailee Biwalkar for helping me with the endeavour. Finally, I'd like to express my gratitude to my friend Siddhant Dongare for perfoming peer review of the questionnaire.

\newpage
\tableofcontents
\newpage
\listoffigures
\newpage
\listoftables

\newpage
\section{Introduction to National Culture}
\subsection{What is Culture?}
\begin{quote}
    Culture is collective programming of the mind which distuinguishes the member of one social group from another.\hfill -Greet Hofstede
\end{quote}

Culture refers to the shared beliefs, values, customs, behaviors, and artifacts that characterize a particular group or society. It includs every aspect of the society be it food, music, religion. The culture can be learned and is passed on generation by generation.

One common theory is that culture developed as a way for early humans to adapt to their environments and survive in challenging conditions. For example, the development of agriculture and animal domestication allowed early humans to settle in one place and build complex societies, which in turn led to the development of language, art, religion, and other cultural practices. Hence group of people having same culture can be identified with how they percieve the environment around them.

Initially, anthropologists believed that culture was a product of biological evolution, and that cultural evolution depended exclusively on physical conditions. Today’s anthropologists no longer believe it is this simple. Neither culture nor biology is solely responsible for the other. They interact in very complex ways, which biological anthropologists will be studying for years to come.

The culture us characterized by few basic elements. The elements of culture are
\begin{itemize}
    \item Symbols: anything which carries a particular meaning recognized by people of same culture
    \item Language: system of symbols with which people communicate
    \item Values: culturally-defined standards that serve as broad guidelines
    \item Beliefs: specific statements that people hold to be true
    \item Norms: rules and expectations by which a society guides the behavior of its members
\end{itemize}

Altough, the culture is very big umbrella term which includes various terms social behaviour, institutions and norms, there have been attempts made to try to analyse the culture. The first attempt in modern times was done by Samuel Pofendorf, \textit{"culture"}, he quoted, \textit{"refers to all the ways in which human beings overcome their original barbarism, and through artifice, become fully human."}

\subsection{National Culture}

National culture refers to the shared beliefs, values, customs, behaviors, and artifacts that characterize a particular country or nation. It is a complex and multifaceted concept that has a significant impact on every aspect of society, including business, education, government, and social interactions. Understanding national culture is important for individuals and organizations that operate in multiple countries or interact with people from diverse cultural backgrounds.

National culture is affected by wide ranging factors such as :
\begin{itemize}
    \item History
    \item Geography
    \item Religion
    \item Language
    \item Political systems
\end{itemize}

As is common knowledge, every country has a unique history. Different countries' histories are unique; some were extremely wealthy, some had limited resources, some frequently experienced natural disasters, and some had fertile territory. The morals and emblems of various nations vary. Different nations have various forms of administration. These variables, which are particular to each country, result in those nations having distinctive cultures.

\subsection{Hofstede's Study on National Culture}

First major attemp at analysing the culture of people according to their nationalities was done by Geert Hofstede. This was carried out in the 1970s. It was a significant research for understanding national culture. Many multinational corporations, who were attempting to expand into new nations as globalisation grew throughout the globe, saw the importance of this study.

Hofstede's cultural dimensions theory \cite{ref:Hofstede's cultural dimensions theory} is a framework for cross-cultural communication, developed by Geert Hofstede. It shows the effects of a society's culture on the values of its members, and how these values relate to behavior, using a structure derived from factor analysis. n 1965 Hofstede founded the personnel research department of IBM Europe (which he managed until 1971). Between 1967 and 1973, he executed a large survey study regarding national values differences across the worldwide subsidiaries of this multinational corporation: he compared the answers of 117,000 IBM matched employees samples on the same attitude survey in different countries. He first focused his research on the 40 largest countries, and then extended it to 50 countries and 3 regions, \textit{"at that time probably the largest matched-sample cross-national database available anywhere."} The theory was one of the first quantifiable theories that could be used to explain observed differences between cultures.

Initially he proposed four dimensions along which cultural values could be analyzed: 
\begin{itemize}
    \item individualism-collectivism: degree to which people in a society are integrated into groups
    \item uncertainity avoidance: a society's tolerance for ambiguity
    \item power distance (strength of social hierarchy): the extent to which the less powerful members of organizations and institutions (like the family) accept and expect that power is distributed unequally
    \item masculinity-femininity (task-orientation versus person-orientation): masculinity is defined as "a preference in society for achievement, heroism, assertiveness, and material rewards for success"
\end{itemize}
Later on, two extra dimensions were added to the list. The fifth one was added by independent researchers in Hong-Kong, long-term orientation and last one was added recently in 2010 by Hofstede, indulgance versus self restraint.

\section{National Culture of India}

\subsection{Introduction}

India's national culture is distinctive. India has been thriving since primordial times. Because of its size, the widespread practise of faith, the availability of food due to its fertile land, and the wide range of climates and landforms throughout the nation, there is a very diverse population. Politically and demographically, this nation was formed as a result of the people's fight against repeated foreign invasions over thousands of years and 250 years of slavery as a colony of an external power 7,000 kilometres away. These all leads to unique characteristics in national culture.

\subsection{Hofstedes Research on National Culture of India}

Hofstede's research on national culture of India\cite{ref:Hofstede's research on national culture of India} has shed light on the complex and diverse cultural landscape of this vibrant nation. By identifying dimensions such as power distance, individualism vs. collectivism, masculinity vs. femininity, uncertainty avoidance, and long-term vs. short-term orientation, Hofstede has provided a framework for understanding the cultural norms and values that shape Indian society.

The result of his cultural analysis can be summarised in Table \ref{Table 1}. We will cover what does individual score mean in this table.

\begin{table}
    \begin{center}
    \begin{tabular}{|c|c|c|}
        \hline
        Index & India & World Average \\
        \hline
        Power Distance & 77 & 56.5 \\
        Long-Term Orientation & 61 & 48 \\
        Masculinity & 56 & 51 \\
        Individualism & 48 & 40 \\
        Uncertainity Avoidance & 40 & 65 \\
        \hline
    \end{tabular}
    \caption{Cultural dimensions scores for India}
    \label{Table 1}
    \end{center}
\end{table}

\subsubsection{Power Distance}

The power distance measures the extenthow much the people with less power are ready to accept unequal power distribution. A high number indicates that the respondents have no trouble following orders from those in higher positions in the hierarchy.

The score of India on this parameter is very high. This indicates that people appreciate vertical hierarchy, acceptance of un-equal rights is high. This results in accessibility in immediate seniority but not so for above layer and loyalty is rewarded for the employee. The power structure tends to be centralized.

\subsubsection{Long-Term Orientation}

This dimension describes how the society sees the past and how much it think the future will be affected by current action. Normative societies, which rank poorly on this metric, favour upholding time-honored customs and standards while being wary of societal change. On the other hand, high-scoring cultures adopt a more practical approach: they promote thrift and efforts in contemporary education as a means of future preparation.

The score of India in this parameter is not tilted significantly to any side. Hence there is no dominant preference among Indians implying different  behaviour for different situation. Altough, Indians  value thrift and perseverence; they have high amount of respect for tradition and they do tend to manipulate things for immediate success.

\subsubsection{Masculinity}

A low score implies that society has feminine characteristics. Hofstede found that in majority of the societies, the characteristics of caring and nurturing are associated with women.

With score of 56, India can be considered a masculine society. He also found that, overall, the India is not equally masculine all characteristics. In terms of visual display of success and power, Indians are highly musciline but in daily life, they also show high humility and abstinence with their ancient culture.

\subsubsection{Individualism}
A high sense of individualism means interpersonal bonding of people is quite loose, and people tend to think about themselves and their immediate families. In the opposite spectrum, in cultures with high collectivism, members in a group can look at other members for unquestioned loyalty.

Again, the score of India is very balanced and not tilted towards any side. This again shows selectivity of the Indians towards few types of groupings. Altough, Indians generally associate themselves with certain groups and assign high status for those in group and those leaving groups feels intense emptiness. Yet religious learning of Indian keep them self oriented with their concept of death and rebirth.

\subsubsection{Uncertainity Aviodance}

Future can never be known but you can make yourself prepared for possible futures. This dimension of cultural analysis tests how much people are willing to tolerate uncertain in future.

Compared to the other world, the India has a very low score. India is traditionally a patient country where tolerance for the unexpected is high; even welcomed as a break from monotony. People generally do not feel driven and compelled to take action-initiatives and comfortably settle into established rolls and routines without questioning.

\section{Assesment of Reflecion of National Culture of India in Government Hospital}

Now that we are up to speed on the theory of national culture of India, we will attempt to evaluate India's national culture in relation to the aforementioned dimension proposed by Hofstede and determine whether we can see the national culture affecting the working of a critical public healtcare offering.

\subsection{Need for Survey}

In order to assess any information, a proper survey needs to be done. Surveys are a valuable tool for collecting data and measuring cultural values and practices. By analysing patterns and trends in the data collected by the survey one can reach good conclusions which forms basis of any study.

\subsection{Choosing the Sample Set}

While choosing a target audience for study, we need to have sample size of good number and quality. In order to study the reflection of national culture in hostpital which typically has working staff of about 80 employee, we need to interview about 20. Also, in any given organisation, the role of employee and his/her vertical position in hierarchy can affect the answers, hence we should have people from multiple categories.

\subsection{Creating a Questionnaire}

A good questionnaire is essential for collecting accurate and relevant data in any research project. A well-designed questionnaire helps to ensure that the information collected is reliable, valid, and free from bias.

The importance of a good questionnaire cannot be overstated, as it is the primary tool for collecting data in many research projects. Without a well-designed questionnaire, the data collected may be inaccurate or incomplete, leading to flawed conclusions and recommendations.

The questionnaire used in this assesment was made by studying book on Organisational Behaviour by Robbins and reading various websites containing national cultures. And it was reviewed by persons from position to provide multiple perspective.

\subsection{Analysing the Survey}

Analyzing the survey is an important step in any research project, as it helps to identify patterns and trends in the data collected. The first step in analyzing the survey is to clean and organize the data, removing any errors or inconsistencies that may have arisen during data collection. This was done with utmost care with help of modern software such as Microsoft Excel and using various graphs and charts.

\section{The Survey}

Using a Google(TM) Form, the project's survey was conducted. Every employee was urged to take part in the survey. In total, 19 responses were given. It was attempted to keep the form from being too lengthy while maintaining the crucial questions in light of the importance of the time of the employees.

\subsection{Composition of Questionnaire}

The questionnaire consisted of 8 section with first section being just write you name and your role in organisation. The secions were made according to the dimension of the culture they are trying to enquire. The question were of different type some being simple factual agree-disagree question, some multiple choice type and some true false type. In many question, option of custom answer was kept available as the given options may not be there. Finally, after each section, an optional question to add any comment was incorporated to make sure to collect any specific input any employee might want to give.


\begin{thebibliography}{99}

    \bibitem{ref:Hofstede's cultural dimensions theory}
    {\em \href{https://en.wikipedia.org/wiki/Hofstede%27s_cultural_dimensions_theory#Dimensions_of_national_culturess/}{Hofstede's cultural dimensions theory}}

    \bibitem{ref:Hofstede's research on national culture of India}
    {\em \href{https://www.hofstede-insights.com/country-comparison/india/}{Hofstede's research on national culture of India}}

    \bibitem{ref: IPCH}
    {\em \href{https://nhm.gov.in/images/pdf/guidelines/iphs/iphs-revised-guidlines-2022/01-SDH_DH_IPHS_Guidelines-2022.pdf}{Indian Public Health Standerds}}

    \bibitem{ref: Survey Form}
    {\em \href{https://drive.google.com/file/d/1fBm0hNM2xrKJiMaiN55BHcJJa5gY-k8D/view?usp=share_link}{Survey Form}}

    \bibitem{ref:Survey Results}
    {\em \href{https://docs.google.com/spreadsheets/d/13JBPbaedCzGmHHtawIFe-Poizsp84CtahEr07WUdKAw/edit?usp=sharing}{Survey of OB of Civil Hospital, Jalna (Responses)}}

\end{thebibliography}

\end{document}