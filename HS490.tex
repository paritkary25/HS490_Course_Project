\documentclass{article}
\usepackage{graphicx}
\usepackage[utf8]{inputenc}
\usepackage{epsfig}
\usepackage{float,graphicx}
\usepackage{listings}
\usepackage{xcolor}
\usepackage{hyperref}
\usepackage{amssymb}
\usepackage{amsmath}
\usepackage{pdfpages}

\definecolor{codegreen}{rgb}{0,0.6,0}
\definecolor{codegray}{rgb}{0.5,0.5,0.5}
\definecolor{codepurple}{rgb}{0,0,0}
\definecolor{backcolour}{rgb}{1,1,1}

\lstdefinestyle{mystyle}{
    backgroundcolor=\color{backcolour},   
    commentstyle=\color{codegreen},
    keywordstyle=\color{black},
    numberstyle=\tiny\color{codegray},
    stringstyle=\color{codepurple},
    basicstyle=\ttfamily\footnotesize,
    breakatwhitespace=false,         
    breaklines=true,                 
    captionpos=b,                    
    keepspaces=true,                 
    numbers=left,                    
    numbersep=5pt,                  
    showspaces=false,                
    showstringspaces=false,
    showtabs=false,                  
    tabsize=2
}

\lstset{style=mystyle}
\title{\textbf{Reflection of National Culture in Civil Hospital, Jalna}\\
HS490: Course Project}

\author{Yash Paritkar\\
Roll Number: 210070096\\ 
\\
Instructor: Prof. Pooja Purang}

\begin{document}
\maketitle

\newpage
\tableofcontents

\newpage
\listoffigures
\newpage
\listoftables

\newpage
\section{Introduction to National Culture}
\subsection{What is Culture?}
\begin{quote}
    Culture is collective programming of the mind which distuinguishes the member of one social group from another.\hfill -Greet Hofstede
\end{quote}

Culture refers to the shared beliefs, values, customs, behaviors, and artifacts that characterize a particular group or society. It includs every aspect of the society be it food, music, religion. The culture can be learned and is passed on generation by generation. 

One common theory is that culture developed as a way for early humans to adapt to their environments and survive in challenging conditions. For example, the development of agriculture and animal domestication allowed early humans to settle in one place and build complex societies, which in turn led to the development of language, art, religion, and other cultural practices. Hence group of people having same culture can be identified with how they percieve the environment around them.
 
\end{document}