\documentclass{article}
\usepackage{graphicx}
\usepackage[utf8]{inputenc}
\usepackage{epsfig}
\usepackage{float,graphicx}
\usepackage{listings}
\usepackage{xcolor}
\usepackage{hyperref}
\usepackage{amssymb}
\usepackage{amsmath}
\usepackage{pdfpages}
\usepackage{indentfirst}

\definecolor{codegreen}{rgb}{0,0.6,0}
\definecolor{codegray}{rgb}{0.5,0.5,0.5}
\definecolor{codepurple}{rgb}{0,0,0}
\definecolor{backcolour}{rgb}{1,1,1}

\hypersetup{
 colorlinks,
 citecolor=blue,
 linkcolor=blue,
 urlcolor=blue}

\lstdefinestyle{mystyle}{
    backgroundcolor=\color{backcolour},
    commentstyle=\color{codegreen},
    keywordstyle=\color{black},
    numberstyle=\tiny\color{codegray},
    stringstyle=\color{codepurple},
    basicstyle=\ttfamily\footnotesize,
    breakatwhitespace=false,
    breaklines=true,
    captionpos=b,
    keepspaces=true,
    numbers=left,
    numbersep=5pt,
    showspaces=false,
    showstringspaces=false,
    showtabs=false,
    tabsize=2
}

\lstset{style=mystyle}
\title{\textbf{Reflection of National Culture in Civil Hospital, Jalna}\\
HS490: Course Project}

\author{Yash Paritkar\\
Roll Number: 210070096\\
\\
Instructor: Prof. Pooja Purang}

\begin{document}
\maketitle

\newpage
\tableofcontents

\newpage
\listoffigures
\newpage
\listoftables

\newpage
\section{Introduction to National Culture}
\subsection{What is Culture?}
\begin{quote}
    Culture is collective programming of the mind which distuinguishes the member of one social group from another.\hfill -Greet Hofstede
\end{quote}

Culture refers to the shared beliefs, values, customs, behaviors, and artifacts that characterize a particular group or society. It includs every aspect of the society be it food, music, religion. The culture can be learned and is passed on generation by generation.

One common theory is that culture developed as a way for early humans to adapt to their environments and survive in challenging conditions. For example, the development of agriculture and animal domestication allowed early humans to settle in one place and build complex societies, which in turn led to the development of language, art, religion, and other cultural practices. Hence group of people having same culture can be identified with how they percieve the environment around them.

Initially, anthropologists believed that culture was a product of biological evolution, and that cultural evolution depended exclusively on physical conditions. Today’s anthropologists no longer believe it is this simple. Neither culture nor biology is solely responsible for the other. They interact in very complex ways, which biological anthropologists will be studying for years to come.

The culture us characterized by few basic elements. The elements of culture are
\begin{itemize}
    \item Symbols: anything which carries a particular meaning recognized by people of same culture
    \item Language: system of symbols with which people communicate
    \item Values: culturally-defined standards that serve as broad guidelines
    \item Beliefs: specific statements that people hold to be true
    \item Norms: rules and expectations by which a society guides the behavior of its members
\end{itemize}

Altough, the culture is very big umbrella term which includes various terms social behaviour, institutions and norms, there have been attempts made to try to analyse the culture. The first attempt in modern times was done by Samuel Pofendorf, \textit{"culture"}, he quoted, \textit{"refers to all the ways in which human beings overcome their original barbarism, and through artifice, become fully human."}

\subsection{National Culture and Hofstede's Study on National Culture}

National culture refers to the shared beliefs, values, customs, behaviors, and artifacts that characterize a particular country or nation. It is a complex and multifaceted concept that has a significant impact on every aspect of society, including business, education, government, and social interactions. Understanding national culture is important for individuals and organizations that operate in multiple countries or interact with people from diverse cultural backgrounds.

National culture is affected by wide ranging factors such as :
\begin{itemize}
    \item History
    \item Geography
    \item Religion
    \item Language
    \item Political systems
\end{itemize}

As is common knowledge, every country has a unique history. Different countries' histories are unique; some were extremely wealthy, some had limited resources, some frequently experienced natural disasters, and some had fertile territory. The morals and emblems of various nations vary. Different nations have various forms of administration. These variables, which are particular to each country, result in those nations having distinctive cultures.

First major attemp at analysing the culture of people according to their nationalities was done by Geert Hofstede. This was carried out in the 1970s. It was a significant research for understanding national culture. Many multinational corporations, who were attempting to expand into new nations as globalisation grew throughout the globe, saw the importance of this study.

Hofstede's cultural dimensions theory \cite{ref:Hofstede's cultural dimensions theory} is a framework for cross-cultural communication, developed by Geert Hofstede. It shows the effects of a society's culture on the values of its members, and how these values relate to behavior, using a structure derived from factor analysis. n 1965 Hofstede founded the personnel research department of IBM Europe (which he managed until 1971). Between 1967 and 1973, he executed a large survey study regarding national values differences across the worldwide subsidiaries of this multinational corporation: he compared the answers of 117,000 IBM matched employees samples on the same attitude survey in different countries. He first focused his research on the 40 largest countries, and then extended it to 50 countries and 3 regions, \textit{"at that time probably the largest matched-sample cross-national database available anywhere."} The theory was one of the first quantifiable theories that could be used to explain observed differences between cultures.

He

\section{National Culture of India}

\begin{thebibliography}{99}

    \bibitem{ref:Hofstede's cultural dimensions theory}
    {\em \href{https://en.wikipedia.org/wiki/Hofstede%27s_cultural_dimensions_theory#Dimensions_of_national_culturess/}{Hofstede's cultural dimensions theory}}

\end{thebibliography}

\end{document}